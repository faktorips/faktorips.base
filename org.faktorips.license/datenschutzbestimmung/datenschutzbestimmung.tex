\documentclass[10pt]{article}
\usepackage{geometry}
\usepackage{cmap}
\usepackage[utf8]{inputenc}
\usepackage{ngerman}
\usepackage[T1]{fontenc}
\usepackage{lmodern}

\geometry{a4paper,left=30mm,right=30mm, top=30mm, bottom=30mm} 

\setlength{\parskip}{1.2ex}
\setlength{\parindent}{0em}

\renewcommand{\familydefault}{cmss}

\makeatletter
\renewcommand{\section}{\@startsection{section}{1}{\z@}{3ex}{1ex}{\normalfont\large\bfseries}}
\renewcommand{\subsection}{\@startsection{subsection}{2}{\z@}{1ex}{1ex}{\normalfont\normalsize\bfseries}}
\makeatother

\begin{document}

\title{Datenschutzbestimmung}
\author{der Faktor Zehn AG, Neumarkter Straße 71, 81673 München}
\date{Stand: 11. Juni 2009}
\maketitle

\section{Geltungsbereich der Datenschutzbestimmungen}

Diese Datenschutzbestimmungen dienen der Information des Nutzers über Art, Umfang und Zwecke der Erhebung und Verwendung
personenbezogener Daten durch die Faktor Zehn AG bei Nutzung des Telemediums www.faktorzehn.org.

\section{Grundsätze und Begriffe}

Die Faktor Zehn AG erhebt und verwendet personenbezogene Daten des Nutzers in Verbindung mit der Nutzung der Website
www.faktorzehn.org der Faktor Zehn AG, der Nutzung des Online-Bereichs der
FaktorZehn Community sowie des Download von Software über die Website der Faktor Zehn AG. Die Erhebung und Verwendung
der personenbezogenen Daten erfolgt nach Maßgabe der anwendbaren gesetzlichen Bestimmungen (insbesondere des
Telemediengesetzes – TMG) sowie nach Maßgabe dieser Datenschutzbestimmungen.

Die Faktor Zehn AG verwendet die erhobenen personenbezogenen Daten für andere Zwecke nur, soweit dies nach den
Bestimmungen des Telemediengesetzes oder durch eine andere Rechtsvorschrift, die sich ausdrücklich auf Telemedien
bezieht, erlaubt ist oder soweit der Nutzer eingewilligt hat.
\begin{description}
\item[Personenbezogene Daten] des Nutzers sind Einzelangaben über dessen persönliche und sachliche Verhältnisse.
Personenbezogene Daten sind Bestandsdaten und Nutzungsdaten.
\item[Nutzer] ist jede natürliche Person, die die Website bzw. den Online-Bereich der FaktorZehn Community nutzt bzw.
die von der Website der Faktor Zehn AG Software herunterlädt.
\item[Bestandsdaten] sind Daten, die für die Begründung, inhaltliche Ausgestaltung oder Änderung des Nutzungsvertrags
über die von Faktor Zehn zum Download zur Verfügung gestellte Software zwischen der Faktor Zehn AG und dem Nutzer
und/oder die im Hinblick auf die Mitgliedschaft in der FaktorZehn Community und der Nutzung des Online-Bereichs der
Community erhoben und verwendet werden (z.B. Angabe von Vor- und Nachname bei einer Online-Registrierung).
\item[Nutzungsdaten] sind Daten, die die Inanspruchnahme der Telemedien ermöglichen (z.B. Angabe von Benutzername und
Passwort bei der Anmeldung für den Online-Bereich der FaktorZehn Community). Zu den Nutzungsdaten zählen insbesondere
Merkmale zur Identifikation des Nutzers sowie Angaben zu Beginn, Ende und Umfangs der jeweiligen Nutzung.
\end{description}

\section{Informationen zu Art, Inhalt und Umfang der Erhebung und Verwendung der Daten}

Die Faktor Zehn AG erhebt und verwendet personenbezogene Daten nur, soweit dies erforderlich ist, die Inanspruchnahme
der Website, die Nutzung des Online-Bereichs der FaktorZehn Community oder den Download von Software zu ermöglichen.

\section{Erhebung und Verwendung von Daten bei der Nutzung des Online-Bereichs der Faktor Zehn Community}

Die Erhebung und Verwendung von personenbezogenen Daten des Nutzers bei Nutzung des Online-Bereichs der FaktorZehn
Community ist im Folgenden beschrieben.

\subsection{Registrierung für die Faktor Zehn Community}

Im Rahmen der Registrierung werden in der Registrierungsmaske personenbezogene Daten des Nutzers abgefragt. Für eine
erfolgreiche Registrierung ist die Eingabe der Pflichtangaben erforderlich. Pflichtangaben sind der vom Nutzer frei zu
wählende Benutzername sowie eine gültige E-Mail-Adresse. Die einzugebende E-Mail-Adresse wird nicht veröffentlicht und
dient nur zur Kommunikation zwischen dem Nutzer und der Faktor Zehn AG. Sofern der Nutzer ausdrücklich in die Zusendung
von E-Mails zu Werbezwecken einwilligt, kann die Faktor Zehn AG dem Nutzer an diese Adresse auch elektronische Werbung
zusenden. Weitere Pflichtangaben sind der Vor- und der Nachname des Nutzers.

\subsection{Benutzername und Passwort}
Der Nutzer wählt bei der Registrierung einen Benutzernamen und ein persönliches Passwort. Diese werden durch die Faktor
Zehn AG per E-Mail an die bei der Registrierung angegebene E-Mail-Adresse bestätigt. Der Nutzer benötigt den
Benutzernamen und das Passwort, um sich bei der FaktorZehn Community anzumelden. Der Nutzer soll das Passwort
vertraulich behandeln. Der Nutzer kann sein Passwort ändern.

\section{Cookies}

Nach der Anmeldung des Nutzers bei der FaktorZehn Community wird grundsätzlich nur für die jeweilige Dauer der Sitzung
ein Cookie auf dem Computer des Nutzers gespeichert. Bei einem Cookie handelt es sich um einen kurzen Eintrag in einem
speziellen Dateiverzeichnis des Computers. Das Cookie dient dazu, den Computer des Nutzers während der Dauer der Sitzung
zu identifizieren.

Beendet der Nutzer die Sitzung mit „Abmelden“, so wird das entsprechende Cookie sofort gelöscht. Beendet der Nutzer die
Sitzung auf andere Weise (z.B. durch Schließen des Browsers), so bleibt das Cookie noch für kurze Zeit auf seinem
Computer gespeichert und wird dann - je nach Einstellung des verwendeten Browsers - gelöscht.

Aktiviert der Nutzer bei der Anmeldung die Option „Angemeldet bleiben“, so wird das Cookie auf dem Computer des Nutzers
bis zu 365 Tage gespeichert. Der Rechner des Nutzers wird dadurch automatisch wiedererkannt. 

\section{Löschung personenbezogener Daten nach Beendigung der Mitgliedschaft in der Faktor Zehn Community}

Bei Beendigung der Mitgliedschaft bei der FaktorZehn Community werden das Konto (Account) des Nutzers, und die
gespeicherten personenbezogenen Daten des Nutzers dauerhaft gelöscht, soweit und solange die Faktor Zehn AG nicht aus
rechtlichen Gründen zur Speicherung dieser Daten verpflichtet ist.

Beiträge und Kommentare, die der Nutzer vor der Beendigung in die Foren der FaktorZehn Community eingestellt hat,
bleiben - soweit die Faktor Zehn AG nicht ihre Löschung vornimmt - nach der Beendigung auch weiterhin anderen Nutzern
der Foren zugänglich.

\section{Log-Dateien}

Bei jedem Download von Software über die Website der Faktor Zehn AG werden Zugriffsdaten des Nutzers auf dem Server der
Faktor Zehn AG in einer Protokolldatei gespeichert (Log-Datei). Hierbei werden die folgenden Daten gespeichert:
\begin{itemize}
 \item IP-Adresse, durch die der Zugriff erfolgt;
 \item Name und IP-Adresse des Rechners, der die Seite anfordert;
 \item Uhrzeit der Sitzung;
 \item übertragene Datenmenge;
 \item Produkt- und Versionsinformationen des vom Nutzer verwendeten Browsers;
 \item Information über das vom Nutzer verwendete Betriebssystem.
\end{itemize}
Die Faktor Zehn AG verwendet die Protokolldaten ohne Zuordnung zur Person des Nutzers nur für statistische Auswertungen
Sie behält sich jedoch vor, die Protokolldaten nachträglich zu überprüfen, wenn aufgrund konkreter Anhaltspunkte der
berechtigte Verdacht einer rechtswidrigen Nutzung besteht.

\section{Änderungen der Datenschutzbestimmungen}

Die Faktor Zehn AG behält sich vor, diese Datenschutzbestimmungen jederzeit zu ändern oder zu ergänzen, sofern dies im
Hinblick auf gesetzliche Anforderungen, wegen Änderungen in der Konzeption und Abläufen der Website, des Online-Bereichs
der FaktorZehn Community bzw. der Download-Optionen oder wegen Änderung einzelner Angebote oder Funktionalitäten
erforderlich ist. Die Faktor Zehn AG wird die entsprechend geänderten Datenschutzbestimmungen auf ihrer Website
veröffentlichen und die Nutzer dort auf die Änderung hinweisen.

\section{Abrufbarkeit der Datenschutzbestimmungen}

Diese Datenschutzbestimmungen können auf der Website der Faktor Zehn AG über das Link ``Datenschutz''
abgerufen und ausgedruckt werden.

\end{document}
